\documentclass[a4paper,10pt]{article}
\usepackage[utf8]{inputenc}
\usepackage{listings}

%opening
\title{Reefmerge}
\author{Grzegorz Konefal}

\begin{document}

\maketitle

\begin{abstract}

This document describes an idea for Master's thesis which is focused on merging problem in Python's git repositories.
Conflicts are more often especially for big and old projects.
A lot of conflicts are simply solvable and it is possible to automatically determine what solution should be applied.
My work is focused on such cases in Python's repositories.

\end{abstract}

\section*{Problem}

I have provided an investigation how big the problem is in open source Python repositories,
which are known as big and mature.
First I have checked how many merge commits in those repositories had conflicts that were resolved manually by developers.
It turns out that averagely 6\% of merge commits had parents that were conflicting
(this number differ from 2\% to 12\% for different repositories).
Of course the number doesn't show how many developer work was put on resolving conflicts
when rebasing or applying patches - unfortunately it is impossible to measure basing on git history,
because rebasing or applying patches leaves no mark in repository's history.

\section*{Solution}

The main idea is to write a tool that automatically resolves git conflicts in Python code,
no matter if conflict occurred while merging, rebasing, or applying a patch.
I expect the tool usage will be most possibly easy,
so this functionality will be injected to git as a plugin - a merge driver.

Lets imagine that programmer A added import of module 're' and programmer B added import of module 'itertools'.
When it comes to merge their code, a conflict occurs - because they commited changes in the same line.
\begin{lstlisting}
<<<<<<<
import itertools
=======
import re
>>>>>>>
\end{lstlisting}
In fact, there are two different import statements, so what is the problem?
Git does not recognize Python syntax and semantics and doesn't understand that this conflict
can be simply resolved just by:
\begin{lstlisting}
import itertools
import re
\end{lstlisting}
We need a tool that understands Python and can resolve such conflicts automatically.

Here I've listed kinds of conflicts in Python code that can be resolved by reefmerge:
\begin{itemize}
\item Conflicts by editing the same line (e.g. added parameter to function)
\item Conflicts by adding different lines in the same place (e.g. import, new function, new variable initialization)
\item New element to dict, list, etc.
\end{itemize}
Adding or modyfing statements is a kind of problem that should not be automatically resolved,
because order of statements matters and it is hard to decide which should be first,
especially when these statements produces side effects that are important from the whole program perspective.

The best option that we could imagine is to somehow tell the git how to resolve conflicts
in Python code. Fortunately, git allows to select custom merging driver.
Reefmerge will be a merge driver that understands Python and resolves conflicts in Python's code automatically.

Of course, there are conflicts that cannot be resolve automatically, for example when someone introduces
syntax errors to code. But I expect that this tool will work in most of cases.

\section*{Alternatives}

There are already available some tools that can help programmer to fix conflicts.
Part of them really attempts to resolve conflicts, the other part deliever some features
that makes fixing conflicts simpler, but doesn't fix conflicts at all.
Reefmerge will attempt to take place of programmer in resolving conflicts in Python code.
But now lets have a look what programmers already have.

\subsection*{git-rerere}
git-rerere is a git tool that is able to remember already fixed conflicts
and its resolutions. If git-rerere remembers some resolution and identical conflict occurred,
git-rerere can automatically resolve this conflict in the same way as it was done previously.

Python conflicts also may reoccur, but if it is a problem like in the previous chapter,
git-rerere will be useless while functions names or implementations differ.
git-rerere is prepared for fixing the same conflict many times rather than smartly resolving new concflicts.

\subsection*{mergetools and difftools}
Conflicts are hard to resolve quickly manually, hence to increase speed of manual conflict resolution
many mergetools was created. These are tools that during manual conflict resolving allows to see e.g. two or three pretty windows
and differents between them, and gives the ability to quickly resolve these conflicts e.g.
by selecting by mouse which version will be applied.

Mergetools helps to increase efficiency of manual conflicts resolving, but they do not attempt
to fix anything by their own. Difftools are similar to mergetools - but they just helps to see
diffs more clearly.

\end{document}
